\documentclass{beamer}

\mode<presentation>
 {\usetheme{CambridgeUS}
  \usecolortheme{beaver}
  \setbeamercovered{transparent}
  % appearance of bullets
  \useinnertheme{rectangles}
  \definecolor{mybullets}{rgb}{0.6,0.0,0.0}
  \setbeamercolor{structure}{fg=mybullets}

  \definecolor{mycolor}{rgb}{0.2,0.2,0.2}
  % bars
  %\setbeamercolor{section in toc}{fg=black,bg=white}
  %\setbeamercolor{alerted text}{fg=mycolor!80!gray}
  %\setbeamercolor*{palette primary}{fg=mycolor!60!black,bg=gray!30!white}
  %\setbeamercolor*{palette secondary}{fg=mycolor!70!black,bg=gray!15!white}
  %\setbeamercolor*{palette tertiary}{bg=mycolor!80!black,fg=gray!10!white}
  %\setbeamercolor*{palette quaternary}{fg=mycolor,bg=gray!5!white}
  %\setbeamercolor*{sidebar}{fg=mycolor,bg=gray!15!white}

  \setbeamercolor*{palette sidebar primary}{fg=mycolor!10!black}
  \setbeamercolor*{palette sidebar secondary}{fg=white}
  \setbeamercolor*{palette sidebar tertiary}{fg=mycolor!50!black}
  \setbeamercolor*{palette sidebar quaternary}{fg=gray!10!white}

  \setbeamercolor{titlelike}{parent=palette primary,fg=mycolor}
  \setbeamercolor{frametitle}{bg=gray!10!white}
  \setbeamercolor{frametitle right}{bg=gray!60!white}

  \setbeamercolor*{separation line}{}
  \setbeamercolor*{fine separation line}{}
}

 \def\Tiny{\fontsize{5pt}{5pt}\selectfont}
 \renewcommand{\arraystretch}{1.2}

% just show sections and subsections in table of contents
\setcounter{tocdepth}{2}

\usepackage[english]{babel}
\usepackage[latin1]{inputenc}

\usepackage{times}
\usepackage[T1]{fontenc}

\usepackage{hyperref}
\usepackage{multicol}

\newcommand{\todo}{\hspace{-2.3pt}$\bullet$ \hspace{5pt}}

\title[Profibus]{\textbf{Pro}cess \textbf{Fi}eld \textbf{Bus}}

\author[Koslowski]{Konstantin Koslowski (316955)}

\institute[]
{TU Berlin \\
 Department of Telecommunication Systems \\
 Telecommunication Networks Group \\
}

\date{June 19th, 2015}

\pgfdeclareimage[height=0.5cm]{university-logo}{tu-tkn-logo.jpg}
\logo{\pgfuseimage{university-logo}}

\begin{document}

\begin{frame}
  \titlepage
\end{frame}

\begin{frame}
\frametitle{Table of Contents}
\setcounter{tocdepth}{1}
\tableofcontents
\end{frame}

\section{Introduction}


\subsection{Development of Profibus}
\begin{frame}
\frametitle{Timeline}
  \textbf{1986}
  \begin{itemize}
    \item Master development plan ``fieldbus'' created in Germany
    \item 21 companies, including Siemens, involved
  \end{itemize}

  \textbf{1989}
  \begin{itemize}
    \item First promoted by \textit{Bundesministerium f\"ur Bildung und
        Forschung (BMBF)}
    \item Goal to implement a bit-serial field bus for factory and process automation
  \end{itemize}

  \textbf{1999}
  \begin{itemize}
    \item Published openly as part of standard \textbf{IEC 61158} \textit{Digital data
        communication for measurement and control - Fieldbus for use in industrial control
        systems}
  \end{itemize}
\end{frame}

\section{System Structure}
\begin{frame}
  \frametitle{System Structure: Introduction}
  \begin{itemize}
    \item \textit{Profibus} is a multi-master system
    \item Operation of multiple systems over a single bus
    \item Three protocols available
      \begin{itemize}
        \item FMS \textit{(field-bus message specification)}
        \item DP \textit{(decentralized peripheral)}
        \item PA \textit{(process automation)}
      \end{itemize}
    \item Devices are categorized in different types
      \begin{itemize}
        \item Masters
        \item Slaves
      \end{itemize}
  \end{itemize}
\end{frame}

\begin{frame}
  \frametitle{System Structure: Layer}
  \textit{Profibus} in the OSI reference model~\cite{profibusmanual}
  \center
  \footnotesize
  \begin{tabular}[h]{l|l|l}
    \textbf{Layer}  & \textbf{Name}     & \textbf{Content} \\
    \hline
    Layer 8         & User Layer        & Profiles \\
    Layer 7         & Application Layer & FMS / DP / PA protocol \\
    Layer 2         & Data Link Layer   & FDL protocol \\
    Layer 1         & Physical Layer    & Transmission Technology
  \end{tabular} \\
\end{frame}


\begin{frame}
  \frametitle{Device Type: Master}
  \begin{itemize}
    \item Active station
    \item Control the data traffic on the bus
    \item When having the \textit{bus access token}: \\
      send messages without external requests
  \end{itemize}
\end{frame}

\begin{frame}
  \frametitle{Device Type: Slave}
  \begin{itemize}
    \item Passive station
    \item No self-initiated bus access
    \item Immediate response to data requested by a master
    \item Can only be controlled by a single master
  \end{itemize}
\end{frame}

\section{Layer 1: Physical Layer}
\begin{frame}
  \frametitle{Physical Layer}
  \begin{itemize}
    \item \textit{Profibus FMS} and \textit{Profibus DP}
      \begin{itemize}
        \item Mostly using RS 485 transmission
        \item Optical transmission via FOC \textit{(fibre optical cable)} possible
      \end{itemize}
    \item \textit{Profibus PA}
      \begin{itemize}
        \item Uses MBP \textit{(Manchester bus powered)}, providing power supply
      \end{itemize}
  \end{itemize}
  \center
  \footnotesize
  \begin{tabular}[h]{l|l}
    \textbf{Type} & \textbf{Transmission technology} \\
    \hline
    0             & copper cable with RS 485 \\
    1             & synchronous MBP \\
    2             & synthetic FOC \\
    3             & glass FOC \\
    4             & HCS FOC
  \end{tabular} \\
  \hfill \\
  \normalsize
  Transmission technology (IRC 61784)~\cite{profibusmanual}
\end{frame}

\begin{frame}
  \frametitle{Physical Layer: RS 485}
  \begin{itemize}
    \item Bus-topology
    \item Twisted-pair cables with $150\Omega$
    \item Data rates from $9.6kbit/s$ to $12Mbit/s$
    \item Distance between repeaters $100m$ to $1200m$
    \item UART coding \\
      \footnotesize
      \begin{itemize}
        \item Start = 0, Parity = EVEN, Stop = 1 \\
        \item
          \begin{tabular}[h]{|c|c|c|c|c|c|c|c|c|c|c|}
            \hline
            Start & Databit 1 & 2 & 3 & 4 & 5 & 6 & 7 & 8 & Parity & Stop \\
            \hline
          \end{tabular}
      \end{itemize}
  \end{itemize}
  Mainly used with \textit{Profibus DP}
\end{frame}

\begin{frame}
  \frametitle{Physical Layer: FOC}
  \begin{itemize}
    \item Star-, bus or ring-topology
    \item Fibre optical cables
    \item Distance between repeaters up to $15km$
  \end{itemize}
\end{frame}

\begin{frame}
  \frametitle{Physical Layer: MBP}
  \begin{itemize}
    \item Bus-topology
    \item Stations are powered through the bus
    \item Safe in explosion-hazardous environments, power can be reduced to a bare minimum
    \item Data rate is fixed to $31.25kbit/s$
    \item Bus length up to $1900m$
    \item Allows branches up to $60m$ to field devices
    \item Manchester coding
  \end{itemize}
\end{frame}

\begin{frame}
  \frametitle{Physical Layer: MBP}
  Manchester coding \\
  \begin{itemize}
    \begin{multicols}{2}
    \item Every bit is coded as a change
      \begin{itemize}
        \item Positive change: ``0''
        \item Negative change: ``1''
        \item Every bit has the same average value
        \item Average used to power the peripherals
        \item Time synchronization possible with every bit
      \end{itemize}
      \columnbreak
      \begin{figure}[h]
        \centering
        \includegraphics[width=140pt]{img/mbp}
        \caption{Manchester coding}
        \label{fig:mbp}
      \end{figure}
    \end{multicols}
  \end{itemize}
  Mainly used with \textit{Profibus PA}
\end{frame}

\section{Layer 2: Data Link Layer}
\begin{frame}
  \frametitle{Data Link Layer}
  The data transmission in \textit{Profibus} is handled by the \textit{fieldbus data link
    (FDL)} layer. \\
  FDL consists of three functions:
  \begin{itemize}
    \item Medium Access Control \textit{(MAC)}
    \item Fieldbus Link Control \textit{(FLC)}
    \item Fieldbus Management \textit{(FMA)}
  \end{itemize}
\end{frame}

\begin{frame}
  \frametitle{Data Link Layer: MAC}
  \begin{itemize}
    \item Make sure only one station transmits data on the bus
    \item When multiple masters are present
      \begin{itemize}
        \item Masters need the access token to send data
        \item Token is cyclically passed via token telegram
        \item To ensure that all master stations can access the bus, token must be passed
          on after a certain timeout
      \end{itemize}
    \item Slaves only respond to requests by a master
  \end{itemize}
  \textit{Profibus FDL} combines master-slave and token passing in a hybrid access principle
\end{frame}

\begin{frame}
  \frametitle{Data Link Layer: FMA}
  Fieldbus Management provides function to manage the layer 1 and 2
    \begin{itemize}
      \item Reset the layers
      \item Set parameters
      \item Get parameters
      \item Inform the user about events or errors
      \item Activate/Deactivate \textit{service access point (SAP)}
    \end{itemize}
\end{frame}

\begin{frame}
  \frametitle{Data Link Layer: Error handling}
  Errors can be caused by
  \begin{itemize}
    \item Faulty transmitters
    \item Badly shielded cables
    \item Signal reflections
    \item Large divergences in time synchronization between stations
  \end{itemize}
  Error rate is smaller than $10^{-4}$ and can be reduced further by error detection and
  correction
\end{frame}

\begin{frame}
  \frametitle{Data Link Layer: Error detection and correction}
  Error Detection
  \begin{itemize}
    \item Hamming distance of 4 by adding a checksum to each packet
    \item At least 4 bits must change to result in an undetected error
    \item This results in \textit{integrity class 2} after standard \textbf{IEC 870-5-1}
  \end{itemize}

  \vspace{10pt}
  \textit{Send Data with No acknowledge (SDN)} service
  \begin{itemize}
    \item Mainly used for synchronization and status messages
    \item The erroneous telegram is discarded
    \item Telegram from the next cycle is used instead
  \end{itemize}
\end{frame}

\begin{frame}
  \frametitle{Data Link Layer: Error detection and correction}
  \textit{Send Data with Acknowledge (SDA)}
  \begin{itemize}
    \item Mainly used between masters, slaves may not always send an acknowledgement
    \item When the sender does not get a response, the telegram is retransmitted
  \end{itemize}

  \vspace{10pt}
  \textit{Send and Request Data (SRD)}
  \begin{itemize}
    \item Service used between masters and slaves
    \item Acknowledgement is packed on top of the data telegram
    \item When the sender does not get a response, the telegram is retransmitted
  \end{itemize}
\end{frame}


\section{Layer 7: Application Layer}
\begin{frame}
  \frametitle{Application Layer: Addressing}
  \begin{itemize}
    \item Every station has a unique address, coded in 1 byte
      \center
      \footnotesize
      \begin{tabular}[h]{l|l}
        \textbf{Address}  & \textbf{Use} \\
        \hline
        $0$               & reserved for tools, e.g.\ programming devices \\
        $1 - n$           & $n$ master stations \\
        $n - 125$         & slave stations \\
        $126$             & reserved as \textit{delivery address} \\
                          & used for changing the address of a slave during runtime \\
        $127$             & reserved as broadcast address
      \end{tabular}
  \end{itemize}
  Components used for the infrastructure, e.g.\ repeaters transmit the data transparently
  and do not require an address \\
\end{frame}

\begin{frame}
  \frametitle{Application Layer: Telegram Formats}
  \begin{itemize}
    \item Without data field \\
      % \footnotesize
      % \begin{tabular}[h]{|c|c|c|c|c|c|}
      %   \hline
      %   SD1 & DA & SA & FC & FCS & ED \\
      %   \hline
      % \end{tabular}
      % \vspace{5pt}
      % \tiny
      % SD1: Delimiter, DA: Destination Address, SA: Source Address, \\
      % FC: Function Code, FCS: Frame Check Sequence, ED: End Delimiter
      % \normalsize
    \item Variable length from $4-249$ byte, payload $1-246$ byte \\
      \footnotesize
      \begin{tabular}[h]{|c|c|c|c|c|c|c|c|c|c|}
        \hline
        SD2 & LE & LEr & SD2 & DA & SA & FC & PDU & FCS & ED \\
        \hline
      \end{tabular} \\
      \vspace{3pt}
      \tiny
      SD2: Delimiter, LE: Length, LEr: Length repeated, DA: Destination Address,
      SA: Source Address, FC: Function Code, \\
      \vspace{-5pt}
      PDU: Protocol Data Unit, FCS: Frame Check Sequence, ED: End Delimiter
      \normalsize
    \item Fixed payload length of $8$ bytes
      % \footnotesize
      % \begin{tabular}[h]{|c|c|c|c|c|c|c|}
      %   \hline
      %   SD3 & DA & SA & FC & PDU & FCS & ED \\
      %   \hline
      % \end{tabular}
      % \normalsize
    \item Token telegram
      % \footnotesize
      % \begin{tabular}[h]{|c|c|c|}
      %   \hline
      %   SD4 & DA & SA \\
      %   \hline
      % \end{tabular}
      % \normalsize
    \item Short telegram
      % \footnotesize
      % \begin{tabular}[h]{|c|}
      %   \hline
      %   SDC \\
      %   \hline
      % \end{tabular}
      % \normalsize
      \hfill \\
    \item Telegram Delimiter, featuring a Hamming distance of 4 \\
      \footnotesize
      \begin{tabular}[h]{|c|c|c|c|c|c|}
        \hline
        SD1   & SD2   & SD3   & SD4   & ED    & SC \\
        \hline
        0x10  & 0x68  & 0xA2  & 0xDC  & 0x16  & 0xE5 \\
        \hline
      \end{tabular}
      \normalsize
  \end{itemize}
\end{frame}

\begin{frame}
  \frametitle{Application Layer: FMS}
  \begin{itemize}
    \item FMS master controls the relationship with FMS slaves
    \item Replaced by \textit{Profibus DP}
  \end{itemize}
\end{frame}

\begin{frame}
  \frametitle{Application Layer: DP}
  \begin{itemize}
    \item \textit{Profibus DP} masters are separated into classes
      \begin{itemize}
        \item Class 1: control a DP system and the slaves assigned, mostly PLC based
        \item Class 2: tool for commissioning, engineering and maintenance, mostly PC
          based
        \item Class 3: clock master, used for time synchronization
      \end{itemize}
      \center
      \begin{figure}
        \includegraphics[width=.50\textwidth]{img/dp_system.png}
        \caption{Structure of a DP system~\cite{profibusmanual}}
      \end{figure}
  \end{itemize}
\end{frame}

\begin{frame}
  \frametitle{Application Layer: DP - Cyclic process data}
  Data exchange between masters and slaves is separated into three
  phases:~\cite{profibusmanual}
  \center
  \footnotesize
  \begin{tabular}[h]{l|l}
    \textbf{Phase}  & \textbf{Action} \\
    \hline
    Diagnosis       & Master requests diagnostic data from slaves \\
    Initialization  & Master sets parameters and checks configuration of slaves \\
    Data Exchange   & Master sends and requests data from the slaves
  \end{tabular} \\
\end{frame}

\begin{frame}
  \frametitle{Application Layer: DP - Data Exchange}
  Class 1 master station:
  \begin{itemize}
    \item Relationship with a slave is called \textit{MS0}
    \item Data exchange is cyclic
    \item Master sends output data to a slave
    \item Slave immediately responds with input data
    \item Master continues with the next slave or restarts the cycle
  \end{itemize}
  \vspace{10pt}
  The minimum cycle time $T_{BCycle}$ can be calculated:
  \begin{align}
    T_{BCycle} = \frac{380 + (N_{Slaves} \cdot 300) + (N_{Bytes} \cdot 11)}{Bitrate} + 75
    \mu s
    \label{minimumcycletime}
  \end{align}
\end{frame}

\begin{frame}
  \frametitle{Application Layer: DP - Data Exchange}
  Class 2 master station:
  \begin{itemize}
    \item Can exist in addition class 1 masters
    \item Can simultaneously be a class 1 master
    \item Relationship with a slave is called \textit{MS1}
    \item Acyclic communication with a slave in an existing MS0 relationship
  \end{itemize}
\end{frame}

\begin{frame}
  \frametitle{Application Layer: PA}
  \begin{itemize}
    \item Uses the same protocol as \textit{Profibus DP}
    \item Can be connected to an existing \textit{Profibus DP} network
      \begin{itemize}
        \item Using a DP/PA coupler
        \item The faster \textit{DP} network is used as a backbone
      \end{itemize}
  \end{itemize}
\end{frame}

% \begin{frame}
%   \frametitle{Transmission Medium}
%   Several kinds of transmission media can be used:
%   \begin{itemize}
%     \item \textbf{Two-wire bus}: Enables differential signal transmission, ensures reliable communication. Requirement for high-speed CAN.
%     \item \textbf{Single-wire bus}: Simpler/cheaper alternative, fall back in case of fault.
%     \item \textbf{Optical transmission medium}: Ensures immunity to electromagnetic noise, used to interconnect different subnets.
%   \end{itemize}
% \end{frame}
%
% \begin{frame}
%   \frametitle{ISO 11898-2: High-speed CAN}
%   \begin{itemize}
%     \item Maximum bit rate 1 Mbps
%     \item Linear bus end-terminated with $120 \Omega$
%     \item Stubs must be shorter than 30 cm
%   \end{itemize}
%   \begin{figure}
% \includegraphics[width=.75\textwidth]{highspeed.png}
% \caption{High Speed CAN Network \cite{iso118982}}
% \end{figure}
% \end{frame}
%
% \begin{frame}
%   \frametitle{ISO 11898-3: Low-speed fault-tolerant CAN}
%   \begin{itemize}
%     \item Maximum bit rate 125 kbps
%     \item Linear or star bus terminated at node with about $100 \Omega$
%     \item Features energy-saving sleep mode
%   \end{itemize}
%   \begin{figure}
% \includegraphics[width=.4\textwidth]{lowspeed.png}
% \caption{Low Speed CAN Network \cite{iso118983}}
% \end{figure}
% \end{frame}
%
% \subsection{Signalling}
% \begin{frame}
%   \frametitle{Bit Encoding}
%   \begin{itemize}
%     \item Level on bus can assume two complementary values:
%     \begin{itemize}
%       \item \textit{dominant}, usually corresponds to logical value 0
%       \item \textit{recessive}, usually corresponds to logical value 1
%     \end{itemize}
%     \item CAN relies on \emph{non-return to zero} (NRZ) bit encoding
%   \end{itemize}
%   \begin{figure}
%   \includegraphics[width=.4\textwidth]{Canbus_levels.png}
% \caption{Levels on the CAN bus \cite{canlevels}}
% \end{figure}
% \end{frame}
%
% \begin{frame}
%   \frametitle{Synchronization}
%   \begin{itemize}
%     \item Timing information extracted from bit stream
%     \item Edges of the signal are used for synchronization
%     \item Bit stuffing to ensure sufficient number of edges
%   \end{itemize}
%     \begin{figure}
%   \includegraphics[width=.6\textwidth]{bitstuffing.pdf}
% \caption{Bit stuffing technique \cite{principles}}
% \end{figure}
% \end{frame}
%
% \section{Data Link Layer}
% \begin{frame}
%   \frametitle{Table of Contents}
%   \tableofcontents[currentsection]
% \end{frame}
% \subsection{Frame Format}
% \begin{frame}
%   \frametitle{Frame Format}
%   \begin{itemize}
%     \item Specifications define standard and extended frame format
%     \begin{itemize}
%       \item Standard: 11 bit identifier
%       \item Extended: 29 bit identifier
%     \end{itemize}
%     \item Standard frame format mostly used
%     \item Identifier describes meaning of message
%     \item Protocol foresees four kinds of frames: \emph{data, remote, error} and \emph{overload}
%   \end{itemize}
% \end{frame}
%
% \begin{frame}
%   \frametitle{Data Frames}
%   \begin{figure}
%     \includegraphics[width=.6\textwidth]{dataframe.pdf}
%     \caption{Format of standard data frames \cite{principles}}
%   \end{figure}
%   \begin{itemize}
%     \item Dominant \emph{start of frame} (SOF) bit
%     \item Arbitration field: identifier and \emph{remote transmission request} bit
%     \item \emph{Data length code} (DLC): Length of data field encoded in 4 bits
%     \item \emph{Cyclic redundancy check} (CRC) encoded in 15 bits
%     \item ACK slot: Recessive at transmitter, dominant at receiver
%     \item \emph{End of frame} (EOF) slot: Seven recessive bits
%   \end{itemize}
% \end{frame}
%
% \begin{frame}
%   \frametitle{Remote Frames}
%   \begin{itemize}
%     \item Generally, source sends out data autonomously
%     \item Protocol allows to poll for data
%     \item Remote format similar to data format
%     \item RTR field is recessive, data frames have higher priority
%     \item Remote frames carry no data
%   \end{itemize}
% \end{frame}
%
% \begin{frame}
%   \frametitle{Error Frames}
%   \begin{itemize}
%     \item Notify nodes that an error has occured
%     \item Consist of two fields:
%     \begin{itemize}
%       \item Error flag: Six dominant/recessive bits.\\
%         $\rightarrow$ Violates bit stuffing rules, error condition is detected
%       \item Error delimiter: 8 recessive bits.
%     \end{itemize}
%     \item Active flag: dominant, transmitted by node in state \emph{error active}
%     \item Passive flag: recessive, transmitted by node in state \emph{error passive}
%   \end{itemize}
% \end{frame}
%
% \begin{frame}
%   \frametitle{Fault Confinement}
%   \begin{itemize}
%     \item Supervises correct operation of MAC sublayer
%     \item Disconnect defective node from bus
%     \item Uses two counters: \emph{transmission} and \emph{receive error count}
%     \begin{itemize}
%       \item On error detect, counter is increased by a given amount
%       \item On success, counter is decreased by one
%       \item Increase amount of detecting node is higher than relying nodes
%     \end{itemize}
%     \item When counter exceeds 127, node switches from error active to error passive
%     \item When counter exceeds 255, node switches to bus off
%   \end{itemize}
% \end{frame}
%
% \begin{frame}
%   \frametitle{Overload Frames}
%   \begin{itemize}
%     \item Used to slow down operations on the bus by adding delays
%     \item Format similar to the error frames
%     \item Hardly ever used because today's CAN controllers are very fast
%   \end{itemize}
% \end{frame}
%
% \subsection{Access Technique}
% \begin{frame}
%   \frametitle{Access Technique}
%   \begin{itemize}
%     \item CAN relies on CSMA for access control:
%     \begin{itemize}
%       \item When no data is exchanged, level on the bus is recessive
%       \item Before transmission, nodes observe the state of the network
%       \item When network is idle, transmission starts immediately
%     \end{itemize}
%     \item Collisions are improbable but not impossible
%     \item CAN introduces collision resolution scheme: \textbf{Bus arbitration}
%   \end{itemize}
% \end{frame}
%
% \begin{frame}
%   \frametitle{CSMA/CR: Bus Arbitration}
%   Bus arbitration essentially identifies the most urgent frame.
%   \vfill
%   \begin{itemize}
%     \item Level on bus is dominant if one node is sending dominant bit
%     \item Nodes can reliably check level on bus
%     \item On transmission, each node compares level on bus against written value
%     \item If node transmits recessive bit but reads dominant, it backs off
%   \end{itemize}
% \end{frame}
%
% \begin{frame}
%   \frametitle{CSMA/CR: Bus Arbitration}
%   \begin{figure}
%     \includegraphics[width=.6\textwidth]{arbitration.pdf}
%     \caption{Arbitration phase in CAN}%Quelle
%   \end{figure}
%
%   \begin{itemize}
%     \item Nodes transmit message identifier starting with MSB
%     \item Lowest identifier corresponds to highest priority
%     \item Message with highest priority wins contention
%   \end{itemize}
% \end{frame}
%
% \subsection{Error Management}
% \begin{frame}
%   \frametitle{Error Management}
%   \begin{itemize}
%     \item Fundamental requirement for CAN is robustness
%     \item Specifications foresee five mechanisms for error detection:
%     \begin{itemize}
%       \item 15 bit wide CRC: Discover up to five erroneous bits
%       \item Frame check: CRC, ACK, EOF delimiters have to be recessive
%       \item Acknowledgement check: Transmitter checks for set ACK bit
%       \item Bit monitoring: Transmitter checks level on bus against written value
%       \item Bit stuffing: Each node verifies if bit stuffing rules have been violated
%     \end{itemize}
%     \item Residual probability for undetected corrupt message is $4.7 \cdot 10^{-11}$ times
%       the frame error rate or less
%   \end{itemize}
% \end{frame}
%
% \subsection{Communication Services}
% \begin{frame}
%   \frametitle{Logical Link Layer}
%   \begin{itemize}
%     \item Sublayer of Data Link Layer
%     \item Provides communication services to higher layers
%     \item Exports only two types of frames:
%     \begin{itemize}
%       \item \textbf{L\_DATA}: Broadcast value over the network
%       \item \textbf{L\_REMOTE}: Ask for value over the network
%     \end{itemize}
%     \item Error and overload frames invisible to higher layers
%     \item Provides \emph{frame acceptance filtering} function
%   \end{itemize}
% \end{frame}
%
% \begin{frame}
%   \frametitle{Frame Acceptance Filtering (FAF)}
%   \begin{itemize}
%     \item Producer transmits information on the bus
%     \item Frame is read by every node in a receive buffer
%     \item FAF determines if information is relevant to the node
%   \end{itemize}
%
%     \begin{figure}
%     \includegraphics[width=.5\textwidth]{filtering.pdf}
%     \caption{Producer/consumer model}%Quelle
%   \end{figure}
% \end{frame}
%
%
% \section{Application Layer}
% \begin{frame}
%   \frametitle{Table of Contents}
%   \tableofcontents[currentsection]
% \end{frame}
% \subsection{CAN-based Application Protocols}
% \begin{frame}
%   \frametitle{Higher Layer Implementations}
%   \begin{itemize}
%     \item CAN specifications do not include application layer tasks
%     \begin{itemize}
%       \item Flow Control
%       \item Device Addressing
%       \item Fragmentation/Defragmentation
%     \end{itemize}
%     \item Several higher layer protocols rely on CAN
%     \item Industrial automation: CANopen, DeviceNet
%     \item Passenger cars: Each manufacturer has its own standard
%   \end{itemize}
% \end{frame}
%
% \section{Summary}
% \begin{frame}
%   \frametitle{Table of Contents}
%   \tableofcontents[currentsection]
% \end{frame}
% \subsection{Main Features of CAN}
% \begin{frame}
%   \frametitle{Advantages \& Disadvantages of CAN}
%   CAN implements a distributed priority-based multi-master communication system.
%   \vfill
%   \begin{itemize}
%     \item Advantages:
%     \begin{itemize}
%       \item Much more simple and robust than token based access schemes
%       \item More flexible than TDMA approaches
%       \item No message will be delayed by lower priority exchanges
%     \end{itemize}
%     \item Drawbacks:
%     \begin{itemize}
%       \item Relatively low maximum throughput
%       \item Bus length limited by bandwidth, arbitration, timing
%       \item Offers no security or authentication schemes
%     \end{itemize}
%   \end{itemize}
% \end{frame}
%
% \section*{Discussion}
% \begin{frame}
%   \frametitle{Thanks for your attention!}
%   \huge{Questions? Ideas? Suggestions?}
% \end{frame}
%
\section*{References}
\begin{frame}[allowframebreaks]
  \frametitle{References}
  \begin{thebibliography}{10}
  \beamertemplatebookbibitems
  \bibitem{profibusmanual}
    Max Felser
    \newblock Profibus Manual: A collection of information explaining PROFIBUS networks
    \newblock http://www.profibus.felser.ch

    \bibitem{profibuswiki}
    Wikipedia
    \newblock Profibus
    \newblock https://en.wikipedia.org/wiki/Profibus

    \bibitem{fieldbuswiki}
    Wikipedia
    \newblock Fieldbus
    \newblock https://en.wikipedia.org/wiki/Fieldbus
%   \bibitem{expanding}
%   Gabriel Leen, Donal Heffernan.
%   \newblock {\em Expanding Automotive Electronic Systems}
%   \beamertemplatearticlebibitems
%     \bibitem{iso118982}
%     EE JRW - Own work.
%     \newblock {\em CAN ISO11898-2 Network}.
%     \newblock Licensed under CC BY-SA 4.0 via Wikimedia Commons, 2015.
%     \newblock \url{http://commons.wikimedia.org/wiki/File:CAN_ISO11898-2_Network.png}
%     \bibitem{iso118983}
%     EE JRW - Own work.
%     \newblock {\em CAN ISO11898-3 Network}.
%     \newblock Licensed under CC BY-SA 4.0 via Wikimedia Commons, 2015.
%     \newblock \url{http://commons.wikimedia.org/wiki/File:CAN_ISO11898-3_Network.png}
%     \bibitem{canlevels}
%     Plupp - Own work.
%     \newblock {\em Canbus levels}.
%     \newblock Licensed under CC BY-SA 3.0 via Wikimedia Commons, 2015.
%     \newblock \url{http://commons.wikimedia.org/wiki/File:Canbus_levels.svg}
   \end{thebibliography}
 \end{frame}
\end{document}
